\documentclass[a4paper,10pt]{article}
\usepackage[]{graphicx}
\usepackage[]{times}
\usepackage{geometry}
\usepackage{framed}
\usepackage[printonlyused,nohyperlinks,nolist]{acronym} % Acronyms

\geometry{verbose,a4paper,tmargin=1.5cm,bmargin=2cm,lmargin=2cm,rmargin=2cm}
\renewcommand{\baselinestretch}{1.2}

% Verbatim example must still be changed manually...
\newcommand{\me}{ffc\relax }

\title{Gici File Format Converter Manual \\ \small (version 2.0)}

\author{
GICI group \vspace{0.1cm} \\
\small Department of Information and Communications Engineering \\
\small Universitat Aut{\`o}noma Barcelona \\
\small http://www.gici.uab.es  -  http://gici.uab.cat/GiciWebPage/downloads.php \\
}

\date{January 2010}

\begin{document}
\maketitle

\section{Description}

This software is a file format converter and much more, including:
\begin{itemize}
 \item File format conversion.
 \item Coefficient Rounding.
 \item Value saturation.
 \item Spatial / Spectral Wavelets.
 \item Cropping.
 \item Spectral DPCM.
 \item Generalized transposition (\textit{i.e.,} component permutations).
\end{itemize}

\section{Requirements}

This software is programmed in Java, so you might need a JAVA Runtime Environment(JRE) to run this application.
We have used SUN JAVA 1.5. 

\begin{description}
\item[JAI] The Java Advanced Imaging (JAI) library is used to load and save images in formats
other than raw or pgm. The JAI library can be freely downloaded from \emph{http://java.sun.com}.
\textbf{Note:} You don't need to have this library installed in order to compile the source code.
\end{description}

\section{Usage}

The application is provided in a single file, a jar file (\emph{dist/\me{}.jar}), that contains the application.
Along with the application, the source code is also provided. If you need to rebuild the jar file, you can use the \texttt{ant} command.

To launch the application you can use the following command: 

\begin{framed}
\texttt{\$ java -Xmx1200m -jar dist/\me{}.jar --help}
\end{framed}

In a GNU/Linux environment you can also use the shell script \texttt{\me{}} situated at the root of the \me{} directory. 

\begin{framed}
\texttt{\$ ./\me{} --help}
\end{framed}

Two examples of usage are provided below:

\begin{itemize}
\item Convert an input big-endian floating-point image into a integer 16bit little-endian image.
\begin{framed}%
\vspace{-1em}%
\begin{verbatim}
$ ./ffc -i "$INFILE-float-bigendian.raw" -ig $Z $Y $X 6 0 \
        -o "$OUTFILE-16bpppb-littleendian.raw" -og $Z $Y $X 3 1
\end{verbatim}%
\vspace{-1em}%
\end{framed}

\item Apply 5 levels of spectral wavelet transform.
\begin{framed}%
\vspace{-1em}%
\begin{verbatim}
$ ./ffc -i "$INFILE-16bpppb-bigendian.raw" -ig $Z $Y $X 3 0 \
        -o "$OUTFILE-16bpppb-bigendian.raw" -og $Z $Y $X 3 0 \
        --spectralWaveletLevel 5
\end{verbatim}%
\vspace{-1em}%
\end{framed}
\end{itemize}

\section{Notes}

If you need further assistance, you might want to contact us directly.

\end{document}
