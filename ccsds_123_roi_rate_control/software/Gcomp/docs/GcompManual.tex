\documentclass[a4paper,10pt]{article}
\usepackage[]{graphicx}
\usepackage[]{times}
\usepackage{geometry}
\usepackage{framed}
\usepackage[printonlyused,nohyperlinks,nolist]{acronym} % Acronyms

\geometry{verbose,a4paper,tmargin=1.5cm,bmargin=2cm,lmargin=2cm,rmargin=2cm}
\renewcommand{\baselinestretch}{1.2}

% Verbatim example must still be changed manually...
\newcommand{\me}{Gcomp\relax }

\title{Gici \me{} manual \\ \small (version 2.0)}

\author{
GICI group \vspace{0.1cm} \\
\small Department of Information and Communications Engineering \\
\small Universitat Aut{\`o}noma Barcelona \\
\small http://www.gici.uab.es  -  http://gici.uab.cat/GiciWebPage/downloads.php \\
}

\date{January 2010}

\begin{document}
\maketitle

\section{Description}

This software is an implementation of common metrics for 2D images and 3D images. The following metrics are included:
\begin{description}
\item[MAE] Mean Absolute Error.
\item[PAE] Peak Absolute Error.
\item[MSE] Mean Squared Error or P-MSE if mask and weights values are defined.
\item[RMSE] Root Mean Squared Error.
\item[ME] Mean Error.
\item[SNR] Signal to Noise Ratio.
\item[PSNR] Peak Signal to Noise Ratio.
\item[PSNR-S] Peak Signal to Noise Ratio computed as in Salomon's book.
\item[SNRVAR] Signal to Noise Ratio calculated with the original image Variance.
\item[EQUALITY] Perfect equality (true/false).
\end{description}

A definition of them can be found in \cite{AS08}.

\section{Requirements}

This software is programmed in Java, so you might need a JAVA Runtime Environment(JRE) to run this application.
We have used SUN JAVA 1.5. 

\begin{description}
\item[JAI] The Java Advanced Imaging (JAI) library is used to load and save images in formats
other than raw or pgm. The JAI library can be freely downloaded from \emph{http://java.sun.com}.
\textbf{Note:} You don't need to have this library installed in order to compile the source code.
\end{description}

\section{Usage}

The application is provided in a single file, a jar file (\emph{dist/\me{}.jar}), that contains the application.
Along with the application, the source code is also provided. If you need to rebuild the jar file, you can use the \texttt{ant} command.

To launch the application you can use the following command: 

\begin{framed}
\texttt{\$ java -Xmx1200m -jar dist/\me{}.jar --help}
\end{framed}

In a GNU/Linux environment you can also use the shell script \texttt{\me{}} situated at the root of the \me{} directory. 

\begin{framed}
\texttt{\$ ./\me{} --help}
\end{framed}

Two examples of usage are provided below:

\begin{itemize}
\item Compare two 3D images using all the metrics and produce a summarized output.
\begin{framed}%
\vspace{-1em}%
\begin{verbatim}
$ ./Gcomp -i1 "$INFILE-16bpppb-bigendian.raw" -ig1 $Z $Y $X 3 0 \
              -i2 "$OUTFILE-16bpppb-bigendian.raw" -ig2 $Z $Y $X 3 0 \
              -f 1
\end{verbatim}%
\vspace{-1em}%
\end{framed}

\item Compare two 2D images using only PSNR.
\begin{framed}%
\vspace{-1em}%
\begin{verbatim}
$ ./Gcomp -i1 "$INFILE-16bpppb-bigendian.raw" -ig1 1 $Y $X 3 0 \
            -i2 "$OUTFILE-16bpppb-bigendian.raw" -ig2 1 $Y $X 3 0 \
            -m 7
\end{verbatim}%
\vspace{-1em}%
\end{framed}
\end{itemize}

\section{Notes}

If you need further assistance, you might want to contact us directly.

\bibliographystyle{IEEEtran}
\bibliography{IEEEabrv,biblio}

\end{document}
