\documentclass[a4paper,10pt]{article}
\usepackage[]{graphicx}
\usepackage[]{times}
\usepackage{geometry}
\usepackage{framed}
\usepackage[printonlyused,nohyperlinks,nolist]{acronym} % Acronyms

\geometry{verbose,a4paper,tmargin=1.5cm,bmargin=2cm,lmargin=2cm,rmargin=2cm}
\renewcommand{\baselinestretch}{1.2}

% Verbatim example must still be changed manually...
\newcommand{\me}{emporda\relax }

\title{Single-Pass Rate Control with Lossless Region of Interest Coding for Pathology Imaging Software Manual \\ \small (version 1.0)}

\author{
GICI group \vspace{0.1cm} \\
\small Department of Information and Communications Engineering \\
\small Universitat Aut{\`o}noma Barcelona \\
\small http://www.gici.uab.es  -  http://gici.uab.cat/GiciWebPage/downloads.php \\
}

\date{July 2020}

\begin{document}
\maketitle

\section{Description}

This software is an implementation of a Single-Pass Rate Control with Lossless Region of Interest Coding employed in a recent research for coding Pathology Imaging. This research in under revision for the IEEE-ACCESS journal  (https://ieeeaccess.ieee.org).

\noindent 
The zip file contains three folders:

\begin{description}
\item[experiments] contains a variety of scripts, which \textbf{run.sh} is the script to be execute to reproduce, partially, the experiments provided in the manuscript.
\item[images] contains the images and the Region of Interest masks used in the experimental results presented. Only images and masks LYMP1, LYMP2, and LYMP3 are provided.
\item[software] folder stores all the software needed to reproduce the experiments. QEmpordaACROI contains the algorithm implementation of the Single-Pass Rate Control with Lossless Region of Interest Coding
\end{description}

\section{Requirements}

This software is programmed in Java, so you might need a JAVA Runtime Environment(JRE) to run this application.
We have used SUN JAVA 1.8. To execute the software some scripts have been provided, thus a bash environment is also required. All this software have been validated in a macOS Catalina v.10.15.5.

\noindent
In case you are interested into recompile the sourcecode we recommend to employ the Apache ant software tool (https://ant.apache.org).

\section{Usage}

The application is provided in QEmpordaACROI within a single file, a jar file (\emph{dist/\me{}.jar}), that contains the application.
Along with the application, the source code is also provided. If you need to rebuild the jar file, you can use the \texttt{ant} command.

To launch the application you can use the following command: 

\begin{framed}
\texttt{\$ java -Xmx1200m -jar dist/\me{}.jar --help}
\end{framed}

In a GNU/Linux environment you can also use the shell script \texttt{\me{}} situated at the root of the \me{} directory. 

\begin{framed}
\texttt{\$ ./\me{} --help}
\end{framed}

To reproduce the experiments partially execute run.sh located in the experiments folder. It will produce the following output:

\begin{framed}
\texttt{\small NCI01.3\_1280\_1280\_1\_0\_8\_0\_0\_1.raw\_rc\_3\_roi\_1\_qstepROI\_1\_ccsds123\_roicoding\_ec\_2\_cm\_1.results
\small 0.067:1.039183:0.83594567:0.20319499:223.0:0.0:223.0:11.839154:Infinity:11.257168:205482
\small 0.133:1.048079:0.83357257:0.21446452:246.0:0.0:246.0:12.676417:Infinity:12.094431:205482
\small NCI01.3\_1280\_1280\_1\_0\_8\_0\_0\_1.raw\_rc\_3\_roi\_1\_qstepROI\_1\_ccsds123\_roicoding\_ec\_2\_cm\_12.results
\small 0.067:0.689606:0.6226139:0.06694987:221.0:0.0:221.0:14.887566:Infinity:14.30558:205482
\small 0.133:0.757404:0.62444496:0.13291667:95.0:0.0:95.0:21.326265:Infinity:20.74428:205482
}
\end{framed}

The description of the first line of the out is:

\begin{description}
\item[NCI01.3\_1280\_1280\_1\_0\_8\_0\_0\_1.raw] is the encoded and decoded image.
\item[rc\_3] the rate control strategy employed. Further details see Coder.java line 562.
\item[qstepROI\_1] the Region of Interest is encoded with a qstep = 1, thus is losslessly recovered.
\item[ec\_2\_cm\_1] indicates the context modelling employed. cm = 1 refers to single context modelling and cm = 12 to full context modelling.
\end{description}

The rest of the lines contains the following information delimited by :.

\begin{description}
\item[output1] is the target bit-rate
\item[output2] is the total bits per sample
\item[output3] is the bits per sample employed for the Region of Interest 
\item[output4] is the bits per sample employed for the non Region of Interest 
\item[output5] is the Peak Absolute Error for the entire image
\item[output6] is the Peak Absolute Error for the Region of Interest
\item[output7] is the Peak Absolute Error for the non Region of Interest
\item[output8] is the Signal Noise Ratio Error for the entire image
\item[output9] is the Signal Noise Ratio Error for the Region of Interest
\item[output10] is the Signal Noise Ratio Error for the non Region of Interest
\item[output11] number of Region of Interest Samples


\end{description}




\section{Notes}

If you need further assistance, you might want to contact us directly.

\bibliographystyle{IEEEtran}
\bibliography{IEEEabrv,biblio}

\end{document}
